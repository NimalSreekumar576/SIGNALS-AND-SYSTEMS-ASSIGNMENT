\let\negmedspace\undefined
\let\negthickspace\undefined
\documentclass[a4,12pt,twocolumn]{IEEEtran}
%\documentclass[conference]{IEEEtran}
%\IEEEoverridecommandlockouts
% The preceding line is only needed to identify funding in the first footnote. If that is unneeded, please comment it out.
\usepackage{cite}
\usepackage{amsmath,amssymb,amsfonts,amsthm}
\usepackage{algorithmic}
\usepackage{graphicx}
\usepackage{textcomp}
\usepackage{xcolor}
\usepackage{txfonts}
\usepackage{listings}
\usepackage{enumitem}
\usepackage{mathtools}
\usepackage{gensymb}
\usepackage[breaklinks=true]{hyperref}
\usepackage{tkz-euclide} % loads  TikZ and tkz-base
\usepackage{listings}
\usepackage{empheq}
%
%\usepackage{setspace}
%\usepackage{gensymb}
%\doublespacing
%\singlespacing

%\usepackage{graphicx}
%\usepackage{amssymb}
%\usepackage{relsize}
%\usepackage[cmex10]{amsmath}
%\usepackage{amsthm}
%\interdisplaylinepenalty=2500
%\savesymbol{iint}
%\usepackage{txfonts}
%\restoresymbol{TXF}{iint}
%\usepackage{wasysym}
%\usepackage{amsthm}
%\usepackage{iithtlc}
%\usepackage{mathrsfs}
%\usepackage{txfonts}
%\usepackage{stfloats}
%\usepackage{bm}
%\usepackage{cite}
%\usepackage{cases}
%\usepackage{subfig}
%\usepackage{xtab}
%\usepackage{longtable}
%\usepackage{multirow}
%\usepackage{algorithm}
%\usepackage{algpseudocode}
%\usepackage{enumitem}
%\usepackage{mathtools}
%\usepackage{tikz}
%\usepackage{circuitikz}
%\usepackage{verbatim}
%\usepackage{tfrupee}
%\usepackage{stmaryrd}
%\usetkzobj{all}
%    \usepackage{color}                                            %%
%    \usepackage{array}                                            %%
%    \usepackage{longtable}                                        %%
%    \usepackage{calc}                                             %%
%    \usepackage{multirow}                                         %%
%    \usepackage{hhline}                                           %%
%    \usepackage{ifthen}                                           %%
  %optionally (for landscape tables embedded in another document): %%
%    \usepackage{lscape}     
%\usepackage{multicol}
%\usepackage{chngcntr}
%\usepackage{enumerate}

%\usepackage{wasysym}
%\newcounter{MYtempeqncnt}
\DeclareMathOperator*{\Res}{Res}
%\renewcommand{\baselinestretch}{2}
\renewcommand\thesection{\arabic{section}}
\renewcommand\thesubsection{\thesection.\arabic{subsection}}
\renewcommand\thesubsubsection{\thesubsection.\arabic{subsubsection}}

\renewcommand\thesectiondis{\arabic{section}}
\renewcommand\thesubsectiondis{\thesectiondis.\arabic{subsection}}
\renewcommand\thesubsubsectiondis{\thesubsectiondis.\arabic{subsubsection}}

% correct bad hyphenation here
\hyphenation{op-tical net-works semi-conduc-tor}
\def\inputGnumericTable{}                                 %%

\lstset{
%language=C,
frame=single, 
breaklines=true,
columns=fullflexible
}
%\lstset{
%language=tex,
%frame=single, 
%breaklines=true
%}

\begin{document}
%


\newtheorem{theorem}{Theorem}[section]
\newtheorem{problem}{Problem}
\newtheorem{proposition}{Proposition}[section]
\newtheorem{lemma}{Lemma}[section]
\newtheorem{corollary}[theorem]{Corollary}
\newtheorem{example}{Example}[section]
\newtheorem{definition}[problem]{Definition}
%\newtheorem{thm}{Theorem}[section] 
%\newtheorem{defn}[thm]{Definition}
%\newtheorem{algorithm}{Algorithm}[section]
%\newtheorem{cor}{Corollary}
\newcommand{\BEQA}{\begin{eqnarray}}
\newcommand{\EEQA}{\end{eqnarray}}
\newcommand{\define}{\stackrel{\triangle}{=}}

\bibliographystyle{IEEEtran}
%\bibliographystyle{ieeetr}


\providecommand{\mbf}{\mathbf}
\providecommand{\pr}[1]{\ensuremath{\Pr\left(#1\right)}}
\providecommand{\qfunc}[1]{\ensuremath{Q\left(#1\right)}}
\providecommand{\sbrak}[1]{\ensuremath{{}\left[#1\right]}}
\providecommand{\lsbrak}[1]{\ensuremath{{}\left[#1\right.}}
\providecommand{\rsbrak}[1]{\ensuremath{{}\left.#1\right]}}
\providecommand{\brak}[1]{\ensuremath{\left(#1\right)}}
\providecommand{\lbrak}[1]{\ensuremath{\left(#1\right.}}
\providecommand{\rbrak}[1]{\ensuremath{\left.#1\right)}}
\providecommand{\cbrak}[1]{\ensuremath{\left\{#1\right\}}}
\providecommand{\lcbrak}[1]{\ensuremath{\left\{#1\right.}}
\providecommand{\rcbrak}[1]{\ensuremath{\left.#1\right\}}}
\theoremstyle{remark}
\newtheorem{rem}{Remark}
\newcommand{\sgn}{\mathop{\mathrm{sgn}}}
%\providecommand{\abs}[1]{\left\vert#1\right\vert}
\providecommand{\res}[1]{\Res\displaylimits_{#1}} 
%\providecommand{\norm}[1]{\left\lVert#1\right\rVert}
%\providecommand{\norm}[1]{\lVert#1\rVert}
\providecommand{\mtx}[1]{\mathbf{#1}}
%\providecommand{\mean}[1]{E\left[ #1 \right]}
\providecommand{\fourier}{\overset{\mathcal{F}}{ \rightleftharpoons}}
%\providecommand{\hilbert}{\overset{\mathcal{H}}{ \rightleftharpoons}}
\providecommand{\system}{\overset{\mathcal{H}}{ \longleftrightarrow}}
	%\newcommand{\solution}[2]{\textbf{Solution:}{#1}}
\newcommand{\solution}{\noindent \textbf{Solution: }}
\newcommand{\cosec}{\,\text{cosec}\,}
\providecommand{\dec}[2]{\ensuremath{\overset{#1}{\underset{#2}{\gtrless}}}}
\newcommand{\myvec}[1]{\ensuremath{\begin{pmatrix}#1\end{pmatrix}}}
\newcommand{\mydet}[1]{\ensuremath{\begin{vmatrix}#1\end{vmatrix}}}
%\numberwithin{equation}{section}
%\numberwithin{equation}{subsection}
%\numberwithin{problem}{section}
%\numberwithin{definition}{section}
%\makeatletter
%\@addtoreset{figure}{problem}
%\makeatother

%\let\StandardTheFigure\thefigure
\let\vec\mathbf

\title{
\Huge\textbf{Discrete Assignment}\\
\Huge\textbf{EE1205} Signals and Systems\\
}
\large\author{Nimal Sreekumar\\EE23BTECH11044}

% make the title area
\maketitle


%\tableofcontents

\bigskip

\renewcommand{\thefigure}{\theenumi}
\renewcommand{\thetable}{\theenumi}
%\renewcommand{\theequation}{\theenumi}


\textbf{Question 11.9.2.5:}
In an A.P., if the \(p\)-th term is \(\frac{1}{q}\) and \(q\)-th term is \(\frac{1}{p}\), prove that the sum of the first \(pq\) terms is \(\frac{1}{2}\brak{pq + 1}\), where \(p \neq q\).(Using Z-transform of x(n)).\\

\solution
\begin{align}
    \frac{1}{q} &= x\brak{0} + pd \\
    \frac{1}{p} &= x\brak{0} + qd
\end{align}

Solving (1) and (2) gives
\begin{align}
    \frac{1}{pq} &= d\\
    x\brak{0} &= 0
\end{align}

\begin{align}
x\brak{n} \xleftrightarrow z  X\brak{z}
\end{align}

\begin{align}
X\brak{z} &=\sum_{n=-\infty}^{\infty} \brak{x\brak{0} + nd} z^{-n}\\
&= x\brak{0} \sum_{n=0}^{\infty} z^{-n} + d \sum_{n=0}^{\infty}nz^{-n}\\
&= \frac{x\brak{0}}{1-z^{-1}} + \frac{dz^{-1}}{\brak{1-z^{-1}}^2}  
\end{align}

Using \brak{3} and \brak{4},

\begin{align}
X\brak{z} &= \frac{z^{-1}}{pq\brak{1-z^{-1}}^2} , \qquad |z| > 1\\
Y\brak{z} &= X\brak{z}U\brak{z} \\
&= \frac{z^{-1}}{pq\brak{1-z^{-1}}^2} \times \frac{1}{\brak{1-z^{-1}}}\\
&= \frac{z^2}{pq(z-1)^3}
\end{align}

Using Contour Integration to find the inverse Z-transform,
\begin{align}
y\brak{n} &= \frac{1}{2\pi j}\oint_C Y\brak{z}z^{n-1}dz \\
&= \frac{1}{2\pi j}\oint_C \frac{z^{n+1}dz}{pq\brak{z-1}^3} \\
&= \frac{1}{\brak{m-1}!} \lim_{z\to\ a}\frac{d^{m-1}}{dz^{m-1}} \brak{\brak{z-a}^m f\brak{z}} 
\end{align}
Where m is the number of repeated poles \brak{\text{here m =3}},

\begin{align}
&= \frac{1}{2!} \lim_{z\to\ 1}\frac{d^2}{dz^2} \brak{\brak{z-1}^3 \frac{z^{n+1}}{pq\brak{z-1}^3}} \\
&= \frac{1}{2} \lim_{z\to\ 1} \brak{n+1} \times n \times \brak{z}^{n-1}\\
y\brak{n} &= \frac{n\brak{n+1}}{2}
\end{align}

\begin{equation}
y\brak{pq} = \frac{pq\brak{pq+1}}{2}
\end{equation}

\begin{tabular}{|c|c|c|}
    \hline
    \textbf{Symbols} & \textbf{Values} & \textbf{Description} \\
    \hline
    $x\brak{n}$ & $\brak{x\brak{0} + nd}\brak{u\brak{n}}$ & general term of the series \\
    \hline
    $y\brak{n}$ & $ \frac{n\brak{n+1}}{2}$ & sum of n terms \\
    \hline
    $y\brak{n}$ & $x\brak{n} * u\brak{n}$ & - \\
    \hline
\end{tabular}\\


\end{document}
