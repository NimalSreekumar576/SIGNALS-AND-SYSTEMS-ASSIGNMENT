\let\negmedspace\undefined
\let\negthickspace\undefined
\documentclass[a4,12pt,twocolumn]{IEEEtran}
%\documentclass[conference]{IEEEtran}
%\IEEEoverridecommandlockouts
% The preceding line is only needed to identify funding in the first footnote. If that is unneeded, please comment it out.
\usepackage{cite}
\usepackage{amsmath,amssymb,amsfonts,amsthm}
\usepackage{algorithmic}
\usepackage{graphicx}
\usepackage{textcomp}
\usepackage{xcolor}
\usepackage{txfonts}
\usepackage{listings}
\usepackage{enumitem}
\usepackage{mathtools}
\usepackage{gensymb}
\usepackage[breaklinks=true]{hyperref}
\usepackage{tkz-euclide} % loads  TikZ and tkz-base
\usepackage{listings}
\usepackage{empheq}
\usepackage[utf8]{inputenc}
\usepackage{pgfplots}

%\usepackage{setspace}
%\usepackage{gensymb}
%\doublespacing
%\singlespacing

%\usepackage{graphicx}
%\usepackage{amssymb}
%\usepackage{relsize}
%\usepackage[cmex10]{amsmath}
%\usepackage{amsthm}
%\interdisplaylinepenalty=2500
%\savesymbol{iint}
%\usepackage{txfonts}
%\restoresymbol{TXF}{iint}
%\usepackage{wasysym}
%\usepackage{amsthm}
%\usepackage{iithtlc}
%\usepackage{mathrsfs}
%\usepackage{txfonts}
%\usepackage{stfloats}
%\usepackage{bm}
%\usepackage{cite}
%\usepackage{cases}
%\usepackage{subfig}
%\usepackage{xtab}
%\usepackage{longtable}
%\usepackage{multirow}
%\usepackage{algorithm}
%\usepackage{algpseudocode}
%\usepackage{enumitem}
%\usepackage{mathtools}
%\usepackage{tikz}
%\usepackage{circuitikz}
%\usepackage{verbatim}
%\usepackage{tfrupee}
%\usepackage{stmaryrd}
%\usetkzobj{all}
%    \usepackage{color}                                            %%
%    \usepackage{array}                                            %%
%    \usepackage{longtable}                                        %%
%    \usepackage{calc}                                             %%
%    \usepackage{multirow}                                         %%
%    \usepackage{hhline}                                           %%
%    \usepackage{ifthen}                                           %%
  %optionally (for landscape tables embedded in another document): %%
%    \usepackage{lscape}     
%\usepackage{multicol}
%\usepackage{chngcntr}
%\usepackage{enumerate}

%\usepackage{wasysym}
%\newcounter{MYtempeqncnt}
\DeclareMathOperator*{\Res}{Res}
%\renewcommand{\baselinestretch}{2}
\renewcommand\thesection{\arabic{section}}
\renewcommand\thesubsection{\thesection.\arabic{subsection}}
\renewcommand\thesubsubsection{\thesubsection.\arabic{subsubsection}}

\renewcommand\thesectiondis{\arabic{section}}
\renewcommand\thesubsectiondis{\thesectiondis.\arabic{subsection}}
\renewcommand\thesubsubsectiondis{\thesubsectiondis.\arabic{subsubsection}}

% correct bad hyphenation here
\hyphenation{op-tical net-works semi-conduc-tor}
\def\inputGnumericTable{}                                 %%

\lstset{
%language=C,
frame=single, 
breaklines=true,
columns=fullflexible
}
%\lstset{
%language=tex,
%frame=single, 
%breaklines=true
%}

\begin{document}
%


\newtheorem{theorem}{Theorem}[section]
\newtheorem{problem}{Problem}
\newtheorem{proposition}{Proposition}[section]
\newtheorem{lemma}{Lemma}[section]
\newtheorem{corollary}[theorem]{Corollary}
\newtheorem{example}{Example}[section]
\newtheorem{definition}[problem]{Definition}
%\newtheorem{thm}{Theorem}[section] 
%\newtheorem{defn}[thm]{Definition}
%\newtheorem{algorithm}{Algorithm}[section]
%\newtheorem{cor}{Corollary}
\newcommand{\BEQA}{\begin{eqnarray}}
\newcommand{\EEQA}{\end{eqnarray}}
\newcommand{\define}{\stackrel{\triangle}{=}}

\bibliographystyle{IEEEtran}
%\bibliographystyle{ieeetr}


\providecommand{\mbf}{\mathbf}
\providecommand{\pr}[1]{\ensuremath{\Pr\left(#1\right)}}
\providecommand{\qfunc}[1]{\ensuremath{Q\left(#1\right)}}
\providecommand{\sbrak}[1]{\ensuremath{{}\left[#1\right]}}
\providecommand{\lsbrak}[1]{\ensuremath{{}\left[#1\right.}}
\providecommand{\rsbrak}[1]{\ensuremath{{}\left.#1\right]}}
\providecommand{\brak}[1]{\ensuremath{\left(#1\right)}}
\providecommand{\lbrak}[1]{\ensuremath{\left(#1\right.}}
\providecommand{\rbrak}[1]{\ensuremath{\left.#1\right)}}
\providecommand{\cbrak}[1]{\ensuremath{\left\{#1\right\}}}
\providecommand{\lcbrak}[1]{\ensuremath{\left\{#1\right.}}
\providecommand{\rcbrak}[1]{\ensuremath{\left.#1\right\}}}
\theoremstyle{remark}
\newtheorem{rem}{Remark}
\newcommand{\sgn}{\mathop{\mathrm{sgn}}}
%\providecommand{\abs}[1]{\left\vert#1\right\vert}
\providecommand{\res}[1]{\Res\displaylimits_{#1}} 
%\providecommand{\norm}[1]{\left\lVert#1\right\rVert}
%\providecommand{\norm}[1]{\lVert#1\rVert}
\providecommand{\mtx}[1]{\mathbf{#1}}
%\providecommand{\mean}[1]{E\left[ #1 \right]}
\providecommand{\fourier}{\overset{\mathcal{F}}{ \rightleftharpoons}}
%\providecommand{\hilbert}{\overset{\mathcal{H}}{ \rightleftharpoons}}
\providecommand{\system}{\overset{\mathcal{H}}{ \longleftrightarrow}}
	%\newcommand{\solution}[2]{\textbf{Solution:}{#1}}
\newcommand{\solution}{\noindent \textbf{Solution: }}
\newcommand{\cosec}{\,\text{cosec}\,}
\providecommand{\dec}[2]{\ensuremath{\overset{#1}{\underset{#2}{\gtrless}}}}
\newcommand{\myvec}[1]{\ensuremath{\begin{pmatrix}#1\end{pmatrix}}}
\newcommand{\mydet}[1]{\ensuremath{\begin{vmatrix}#1\end{vmatrix}}}
%\numberwithin{equation}{section}
%\numberwithin{equation}{subsection}
%\numberwithin{problem}{section}
%\numberwithin{definition}{section}
%\makeatletter
%\@addtoreset{figure}{problem}
%\makeatother

%\let\StandardTheFigure\thefigure
\let\vec\mathbf

\title{
\Huge\textbf{Gate EE 2023}\\
\Huge\textbf{EE1205} Signals and Systems\\
}
\large\author{Nimal Sreekumar\\EE23BTECH11044}

% make the title area
\maketitle


%\tableofcontents

\bigskip

\renewcommand{\thefigure}{\theenumi}
\renewcommand{\thetable}{\theenumi}
%\renewcommand{\theequation}{\theenumi}


\textbf{Question Gate 2023 EE:}
For the signals x\brak{t} and y\brak{t} shown in the figure, $z\brak{t}=x\brak{t}*y\brak{t}$ is maximum at $t=T_1$. Then $T_1$ in seconds is .......... \brak{\text{Round off to the nearest integer}}\\

\begin{tikzpicture}
\begin{axis}[xmin=-3, xmax=7, ymin=-3, ymax=3, axis lines=middle, xlabel={$x\brak{t}$}, title={y(t)}]
 \addplot[blue] coordinates {(-3,0) (1,0)};
  \addplot[blue] coordinates {(1,0) (1,-2)};
   \addplot[dashed] coordinates {(0,-2) (1,-2)};
  \addplot[blue, domain=1:5] {x - 3};
  \addplot[blue] coordinates {(5,2) (5,0)};
  \addplot[blue] coordinates {(5,0) (7,0)};
  \addplot[dashed] coordinates {(0,2) (5,2)};
    \end{axis}
\end{tikzpicture}


\begin{tikzpicture}
    \begin{axis}[xmin=-3, xmax=3, ymin=-3, ymax=3, axis lines=middle, xlabel={$t$} ,title={$x\brak{t}$}]
        \addplot[blue] coordinates {(-3,0) (-1,0)};
        \addplot[blue] coordinates {(-1,0) (-1,1)};
        \addplot[blue] coordinates {(-1,1) (1,1)};
        \addplot[blue] coordinates {(1,1) (1,0)};
        \addplot[blue] coordinates {(1,0) (3,0)};
    \end{axis}
\end{tikzpicture}

\solution
\begin{align}
z\brak{t} &=x\brak{t}*y\brak{t} = y\brak{t}*x\brak{t}\\
z\brak{t} &=\int_{-\infty}^{\infty} y\brak{\tau}x\brak{t-\tau}d\tau
\end{align}

$x\brak{\tau}$ is an even signal,
\begin{align}
x\brak{\tau}= x\brak{-\tau}\\
 x\brak{-\tau}= 
    \begin{cases}
        1 & ; -1\leq -\tau \leq 1 \\
        0 & ; \text{otherwise} \\
    \end{cases}
    \end{align}
    
    \begin{align}
    x\brak{-\tau} \xleftrightarrow{\text{Time shifting}} x\brak{t-\tau}\\
    x\brak{t-\tau}= 
    \begin{cases}
        1 & ; t-1\leq t-\tau \leq t+1 \\
        0 & ; \text{otherwise} \\
    \end{cases}
\end{align}\\

For $z\brak{t}$ to be maximum both $y\brak{\tau}$ and $x\brak{t-\tau}$ must be maximum,
\begin{align}
\implies t-1 &= 3 \quad \text{or} \quad t+1 = 5 \nonumber \\
t &= T_1 = 4 \nonumber
\end{align}

\end{document}
